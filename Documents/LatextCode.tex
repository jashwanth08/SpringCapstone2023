\documentclass{article}
\usepackage{graphicx} % Required for inserting images

\title{Prediction of impact on GDP and Unemployment due to COVID19}
\author{Jashwanth Reddy Baggari,
        Jyothi Vasamsetty,
        Nikhil Velakurthy}
\date{February 2023}

\begin{document}

\maketitle

\section{Introduction}
    
\paragraph{Covid 19 is spreading worldwide. Scientists have debated the outbreak’s cause.
The World Health Organization estimates that over a quarter million people had died from the illness 
by May 27, 2020. Within months, this virus had infected millions of individuals worldwide. Many facets 
of the covid problem exist in one country. Countries around the world have taken steps like staying 
indoors, social isolation, hand washing, travel restrictions, lockdowns, and more to stop the spread 
of the disease. Lockdowns, for example, are extreme, have unparalleled effects on daily life, and have 
major economic consequences. The recent global lockdowns have had a major influence on global GDP, 
making precise forecasting of COVID-19-related aspects even more important.} 
    
\paragraph{The lockout stretched the supply chain and made streamlining crucial products opaque. The informal 
sector and hourly workers are especially at danger. Uncertainty plagues many US food farmers. Hotels 
and airlines are laying off workers and lowering salaries. We evaluate the spread under the economy’s 
effect using machine learning algorithms with linear regression models. Predicting future trends by 
analyzing past or present data. Such analysis and forecasting involve defining the job, acquiring 
relevant data from many sources, assessing the data, undertaking statistical analysis, constructing a 
data model, deploying the collected data using numerous approaches, and monitoring the model.}

\paragraph{Forecasting has been one of the most successful statistical methods for detecting and 
analyzing patterns and predicting future events, allowing for early and mitigating action. In this paper, we use 
a linear regression model and clustering to examine the data on COVID-19 cases and the pandemic’s 
economic impact. We also provide many data visualizations. We evaluate, clean, load, characterize, 
partition, apply the model, and visualize the data in this procedure.}

\section{Literature Survey}
    
\paragraph{ Our study project used these three papers and publications.
It aids economic studies on COVID-19.} 
    
\paragraph{ In the first study, they seek a correlation between socio-economic 
factors and spread in each affected country. The key
contribution is projecting spread using categorization models
based on over 1000 socioeconomic parameters and 190 nations. Given constructed 
classifiers, an importance analysis is 
suggested to identify the most important classification indication}

\paragraph{ The second paper examined COVID-19’s effects on elementary, secondary, and tertiary sectors. The COVID 19 outbreak
reaction is unthinkable. A quarter of the world’s population is
under lockdown as governments declare emergencies, gather
cash, and ask residents to change their behaviour to stop
the epidemic. With a new recession and economic crisis,
health, industry, government, and community policy making
is crucial. A comprehensive social-economic development
strategy that promotes businesses with stable and sustainable
business models through infrastructure and sector-by-sector
programmes.}

\paragraph{The third paper examines the influence of COVID-19 on unemployment rates in 13 European nations from January 2019 to August 2020 using a panel data technique. The report also examines the causes that contributed to the pandemic's surge in unemployment rates. According to the survey, the closure of non-essential enterprises such as restaurants and retail establishments has been a key contributor to the rise in unemployment rates. The study also indicates that government interventions, such as vacation plans and unemployment compensation, have helped reduce the impact of the epidemic on unemployment rates.}


\paragraph{ In our paper, we used two data sets covid related unemployment 
and country population statistics to assess the pandemic’s economic impact. 
We utilized Linear Regression and Clustering to determine how population drop 
and unemployment during COVID-19 affected global GDP.
}

\section{References:}

\begin{itemize}
  \item Senthil Kumar, Mohan , John A , Ahed Abugabah, Adimoolam M,
Shubham Kumar Singh, Ali Kashif Bashir, Louis Sanzogni. An approach
to forecast impact of Covid-19 using supervised machine learning
model,2021
  \item R. Kumari et al., ”Analysis and predictions of spread, recovery, and
death caused by COVID-19 in India,” in Big Data Mining and Analytics,
vol. 4, no. 2, pp. 65-75, June 2021, doi: 10.26599/BDMA.2020.9020013.
  \item Su, C.W., Dai, K., Ullah, S. and Andlib, Z., 2022. COVID-19 pandemic and unemployment dynamics in European economies. Economic Research-Ekonomska Istraživanja, 35(1), pp.1752-1764.
\end{itemize}

\end{document}
